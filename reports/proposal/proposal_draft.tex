\documentclass[10pt]{article}
\usepackage{graphicx}
\usepackage{geometry}
\usepackage{hyperref} 
\geometry{a4paper,left=2.5cm,right=2.5cm,top=2cm,bottom=2cm}

\begin{document}

\title{Project Proposal for Data1030}
\author{Yue Zhuang}

\maketitle

\begin{abstract}
In this project, we would like to dig into the popularity of news in multiple 
social media platforms. With the timing of publishment an important feature 
that affects the popularity of news, we will develop a real time predicting 
system for certain piece of news. The result of popularity prediction will 
be helpful for the existing news recommend system.
\end{abstract}

\section{Problem description}
In this section, we will describe the problem we want to solve.
\subsection{traget variables}
The target variable of our project would be the popularity rate of a certain 
piece of news.
\subsection{problem type}
The problem would be regression since we will predict the popularity rate of news.
\subsection{interest \& importance}
Unlike most regression problems which has well organized features as the input 
of a ML model, the popularity rate of a piece of news is effected by both the 
content of the news and the timing of publishment. We will need to dig into the 
time series data to investigate the popularity trend of a certain topic. The way 
we select features for the trend model will greatly affect the performance of 
our prediction model, bringing challenge to the problem.
\\\\
The prediction of popularity rate of news can be of significant importance for the 
recommend systems. Social media platforms will be able to push the most popular 
piece of news to consumers by introducing the real-time prediction model to their 
existing recommend system. 


\section{Dataset description}
In this section, we will describe the dataset -- News Popularity in Multiple 
Social Media Platforms Data Set. This is a large data set of news items and 
their respective social feedback on multiple platforms, including: Facebook, 
Google+ and LinkedIn. The collected data relates to a period of 8 months, 
between November 2015 and July 2016, accounting for about 100,000 news items 
on four different topics: economy, microsoft, obama and palestine.

\subsection{dataset size}
We have multiple csv files storing data in two distinct formats.

\subsubsection{news final sheet}
In this csv file, each row represents a certain piece of news, with total 93240 
rows.\\\\ The columns includes 11 features: its IDLink, title, headline, source, 
topic, publishdate, sentimenttitle, sentimentheadline and the final popularity 
on different social media platforms (Facebook, GooglePlus, LinkedIn).

\subsubsection{times series sheet}
In these csv files, each file includes data for the time series popularity rate 
for news on a certain social media platform and with certain topics.\\\\
Each row represents a certain piece of news, with approximatel 3000 rows in each 
csv file. The columns are the popularity rates in each time slice, with a total of 
144 slices. The time series starts from the time of publishment and each time slice 
has a range of 20 minutes.

\subsection{documentation}

\subsubsection{news final sheet}

\begin{table}[h!]
    \begin{center}
      \caption{News Final Sheet}
      \label{tab:table1}
      \begin{tabular}{l|c|r} 
        \textbf{Column Name} & \textbf{Type} & \textbf{Brief Description}\\
        \hline
        IDLink & numerical & Unique identifier of news items\\
        Title & string & Title of the news item according to the official media sources\\
        HeadLine & string & Headline of the news item according to the official media sources\\
        Source & categorical & Original news outlet that published the news item\\
        Topic & categorical & Query topic used to obtain the items in the official media sources\\
        Publishdate & numerical & Date and time of the news items' publication\\
        SentimentTitle & numerical & Sentiment score of the text in the news items' title\\
        SentimentHeadline & numerical & Sentiment score of the text in the news items' headline\\
        FaceBook & numerical & Final value of the popularity according to Facebook\\
        GooglePlus & numerical & Final value of the popularity according to Google+\\
        LinkedIn & numerical & Final value of the popularity according to LinkedIn\\
    \end{tabular}
  \end{center}
\end{table}


\subsubsection{times series sheet}
\begin{table}[h!]
    \begin{center}
      \caption{Time Series Sheet}
      \label{tab:table2}
      \begin{tabular}{l|c|r} 
        \textbf{Column Name} & \textbf{Type} & \textbf{Brief Description}\\
        \hline
        IDLink & numerical & Unique identifier of news items\\
        TS1 & numerical & Level of popularity in time slice 1 (0-20 minutes upon publication)\\
        TS2 & numerical & Level of popularity in time slice 2 (20-40 minutes upon publication)\\
        ... & ... & ...\\
        TS144 & numerical & Final level of popularity after 2 days upon publication\\
      \end{tabular}
    \end{center}
  \end{table}

\subsection{public projects}
In this section, we will give short descriptions for 2 public projects where 
the data has been used, and talk about how the features were used.
\subsubsection{public project 1}
\textbf{Multi Source Social Feedback of Online News Feeds}\\
Author: Nuno Moniz, Luis Torgo\\
Date: Jan 2018\\
\\
In this project, the authors provided a smoothed approximation of the amount of 
news per day, for each topic from both Google News and Yahoo. They also 
concerns the evolution of news items’ popularity in the various social media sources, 
finding that news items obtain close to half of their final popularity in a 
short amount of time. The authors used R to analyze the evolution of available 
information in each time slice for all topics and social media sources, coming to 
statistical results.\\

\subsubsection{public project 2}
\textbf{The Utility Problem of Web Content Popularity Prediction}\\
Author: Nuno Moniz, Luis Torgo\\
Date: May 2018\\
\\
In this project, the authors provided a study of numerical web content popularity 
prediction approaches and their ability to forecast and rank highly popular web content.
Kernel regression and k nearest neighbour methods are used to timely predict the popularity 
of a certain piece of news and to provide a ranking for those news. The authors did not 
release which features they have chosen for the model.\\

\subsubsection{our project}
In our project, we will focus on predicting the popularity for each piece of news on a 
real-time basis. We will try multiple advanced machine learning models and pay more effort
to data preprocessing and feature selection. What's more, our training will focus on the most popular 
pieces of news rather than all available news. 

\section{Data Preprocessing}

\begin{table}[h!]
    \begin{center}
      \caption{Data Preprocessing for News Final Sheet}
      \label{tab:table3}
      \begin{tabular}{l|c|c|c} 
        \textbf{Column (Feature) Name} & \textbf{Type} & \textbf{methods} & \textbf{reasoning}\\
        \hline
        Source & categorical & OneHotEncoder & categorical data with no specific order\\
        Topic & categorical & OneHotEncoder & categorical data with no specific order\\
        Publishdate & numerical &  & used later to get access to another sheet \\
        SentimentTitle & numerical & StandardScaler & continuous data without exact range\\
        SentimentHeadline & numerical & StandardScaler & continuous data without exact range\\
      \end{tabular}
    \end{center}
\end{table}


  \begin{table}[h!]
    \begin{center}
      \caption{Data Preprocessing for News Final Sheet}
      \label{tab:table3}
      \begin{tabular}{l|c|c|c} 
        \textbf{Column (Feature) Name} & \textbf{Type} & \textbf{methods} & \textbf{reasoning}\\
        \hline
        TSi & numerical & Devide by the final popularity TS144 & get percentage data\\
        
        TS144 & numerical & / & final level of popularity\\
      
      \end{tabular}
    \end{center}
\end{table}

Now we have 5 features for sheet1 and 144 features for sheet2. We may make extra progress to get 
those two sheets connected, providing additional convincing features for our model.
\\\\
Github url: 
\href{https://github.com/zysophia/News_Rank_Prediction}{https://github.com/zysophia/News\_Rank\_Prediction}.
\end{document}




